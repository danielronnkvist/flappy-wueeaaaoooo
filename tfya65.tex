\documentclass[11pt, titlepage, twoside, a4paper]{article}
\usepackage[utf8]{inputenc}

\begin{document}
\title{Flappa Freqt}
\author{Cecilia Lagerwall - cecla125 \\ Daniel Rönnkvist - danro716 \\ 
Therese Komstadius - theko867 \\ \\ Linköpings universitet}
\date{2014-10-10}
\maketitle
\begin{abstract}
flappa för fan
\end{abstract}
\section{Inledning}
I denna rapport redovisas projektet “Flappa Freqt” som genomfördes i kursen Ljudfysik, TFYA65, ht 2014 vid Linköpings universitet. Syftet med projektet var att skapa ett spel där spelaren kontrollerade spelet med hjälp av rösten.
\section{Bakgrund}
Eftersom projektet skulle avhandla ljudfysik blev det tidigt bestämt att fokus skulle ligga på att uppta och omvandla ljud. Därför baserades grundiden för spelet på ett redan existerande koncept. Konceptet som valdes var det bakom spelet “Flappy bird”, med skillnaden att olika röstfrekvenser skulle förlytta spelobjektet i vertikalled. Det beslutades även att spelet skulle utvecklas för en enhet med inbyggd mikrofon, därför valdes Javascript som programmeringsspråk för att utveckla till en webbläsare. Javascript valdes även för att det skulle vara enklare att hitta ett färdigt spelskelett att utgå ifrån för att inte lägga tid på att utveckla spelet.
\section{Metod}
Hello world!
\end{document}
